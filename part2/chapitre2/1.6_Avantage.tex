\subsection{Avantage de HBase}
Les avantages de HBase sont nombreux, en plus de ceux que nous avons énoncé dans
le chapitre précédent nous citons :

\begin{itemize}[label=\textbullet]

\item \textbf{Assure l'extensibilité et la calculabilité.} 

\item \textbf{Importante tolérance aux pannes :} Les données sont répliquées sur les différents serveurs du Data Center, et un failover (basculement) automatique garantit une haute disponibilité.

\item \textbf{Rapidité:} Il propose des lookups en temps réel ou presque, et un processing server side \footnote{\textbf{Server-side:} L'expression server-side (coté serveur) fait référence à des opérations qui sont effectuées par le serveur dans la communication entre client et serveur dans un réseau informatique.} par le biais de filtres et de coprocesseurs. Un caching in-memory est également proposé.

\item \textbf{Distribution transparente de la donnée :} Répartition de la charge est faite par le système lui même.

\item \textbf{Les langages de programmation supportés :} Hbase supporte beaucoup de langage de programmation : C, C\#, C++, Groovy, Java, PHP, Python, Scala.

\item \textbf{Utile pour les données de capteurs:} HBase est très adapté aux données collectées de façon incrémentielle a partir de diverses sources. Cela inclut l'analytique sociale, les scieries chronologiques, la mise a jour des tableaux de bord interactifs avec les tendances et les compteurs, et la gestion de systèmes de journal d'audit.

\end{itemize}