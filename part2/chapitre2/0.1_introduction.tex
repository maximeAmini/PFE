\section*{Introduction}
Face à l’essor du Big Data, de nombreuses entreprises doivent désormais traiter des quantités massives de données. Dans ce contexte, les bases de données d’autrefois ne peuvent plus répondre à leurs besoins. Le volume de données à traiter est tel que les requêtes excèdent bien souvent les capacités d’un seul serveur. De même, les bases de données SGBDR ne peuvent prendre en charge des quantités massives de lectures et d’écritures.

Pour dépasser ces limites, de nouveaux SGBD dit "NoSQL" ont vu le jour. La particularité de ceux-ci est qu'ils n'imposent pas de structure particulière aux données, ils relâchent les contraintes qui empêchent les SGBDR de distribuer le stockage des données et sont linéairement scalables. HBase fait partie de cette catégorie de SGBD.

Nous présenterons dans ce chapitre Apache Hbase afin de montrer ses avantages et inconvénients et la façon dont ils gère cette masse de données qu'est le big data.