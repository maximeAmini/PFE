\subsection{Enjeux par rapport à l'énergie:}

À l’heure actuelle, les réseaux électriques doivent faire face à de nouveaux besoins en énergie, avec notamment le développement de la climatisation, des appareils électroniques ou du chauffage électrique. Cette hausse devrait être amplifiée par de nouveaux usages tels que la voiture électrique ou les pompes à chaleur. Les smart grids visent à apporter une réponse à ces contraintes.

\begin{enumerate}
\item \textbf{Des avantages économiques et environnementaux :}

Les smart grids améliorent la sécurité des réseaux électriques. En équilibrant l’offre et la demande, ils évitent le suréquipement des moyens de production et permettent une utilisation plus adaptée des moyens de stockage de l’électricité, disponibles de manière limitée.

Les réseaux intelligents augmentent aussi l’efficacité énergétique globale : ils réduisent les pics de consommation, ce qui atténue les risques de panne généralisée.
Enfin, ils limitent l’impact environnemental de la production d’électricité en réduisant les pertes et en intégrant mieux les énergies renouvelables.

\item \textbf{Les limites dans la mise en œuvre :}

Cependant, le coût des investissements reste élevé. En effet, les smarts grids doivent être implantés sur l’ensemble du réseau et impliquer tous les acteurs pour être efficaces.

L’autre obstacle est la diversité des acteurs, car ils doivent mettre au point des systèmes communicants variés avec des logiques convergentes. De plus, les données recueillies sont complexes à gérer et à stocker, compte tenu de l’importante quantité d’informations à traiter. 

Enfin, les informations sur les horaires ou les activités des consommateurs et des producteurs  sont confidentielles.  Des normes sur la protection des données doivent être appliquées. 
\end{enumerate}