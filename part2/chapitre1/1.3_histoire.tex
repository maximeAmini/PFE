\subsection{Passé et présent:}

Apparue dans les années 1980, la lecture automatique des compteurs (pour surveiller les charges électriques chez le consommateur) est une première étape dans l’émergence des smart grids.
Elle évolue dans les années 1990 vers le principe du compteur communicant, qui renseigne sur la variation de consommation électrique au cours de la journée. En 2000, le projet italien Telegestore est le premier exemple de smart grid. Par l’intermédiaire de ces compteurs, il relie au réseau un grand nombre de foyers (27 millions).

Le suivi et la synchronisation des réseaux sont été améliorés dans les années 1990 par la mise en place de capteurs analysant rapidement et à longue distance les anomalies électriques. Le premier système de mesure utilisant ce type de capteurs est opérationnel en 2000 aux États-Unis.
Aujourd’hui, les réseaux intelligents se développent progressivement. L’expression smart grids se généralise en 2005 avec la mise en place par la Commission Européenne de la plateforme technologique « Smartgrids ».

Les préoccupations environnementales et les attentes concernant la continuité de la fourniture d’électricité contribuent au déploiement de cette technologie. Les nombreux blackouts, notamment aux États-Unis ou en Italie, rappellent le besoin de moderniser des réseaux électriques très vieillissants.

Actuellement, malgré l’engouement des pouvoirs publics et des industriels, les implantations restent locales et parfois expérimentales. Le développement est progressif et l’adaptation des infrastructures prend du temps. En définitive, le développement des smart grids relève davantage d’une évolution dans l’optimisation des réseaux que d’une révolution technologique.