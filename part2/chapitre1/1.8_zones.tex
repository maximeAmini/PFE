\subsection{Zone de présence ou d'application :}
À court et moyen termes, les réseaux intelligents seront essentiellement déployés dans les pays développés car la modernisation du réseau nécessite d’importants investissements 

Les États-Unis ont été précurseurs dans le développement des smart grids. De grands investissements sont en effet consentis afin de moderniser un réseau électrique défaillant et souvent obsolète.

En Europe, le niveau des avancées varie selon les pays. Les pays dont les réseaux sont fragiles et dont la production est largement émettrice de CO2 sont les plus volontaires (comme l’Italie et l’Espagne), tout comme ceux qui ont des préoccupations écologiques anciennes comme la Suède. La France dont le réseau est plutôt de bonne qualité et dont le parc de production est peu émetteur de CO2 affiche des objectifs plus lents.

Les investissements entrepris et les avancées réalisées concernent principalement l’installation de compteurs intelligents à ce jour. On estime à 80 \% le nombre de foyers qui pourraient théoriquement être équipés de compteurs intelligents d’ici 2020 en Europe. Il s’agit d’une condition indispensable mais non suffisante pour avoir des réseaux intelligents réellement efficaces. L’effort devra être conduit en parallèle sur les autres composants du réseau, notamment son système d’information.

Pour ce qui est de l'algérie Sonelgaz a développé la télérelève avec la généralisation des compteurs intelligents (smart meters) pour les clients en HTA.

\subsection{Futur des Smart Grid}
À long terme, le développement des smart grids devrait s’étendre à l’ensemble des réseaux interconnectés.

Toutefois, l’implantation des réseaux intelligents dépend de l’efficacité des dispositifs techniques et de l’implication des parties prenantes.

Parmi elles, les consommateurs auront un rôle clé. En effet, l’équilibre du système électrique sera davantage géré par l’utilisateur final. Une sensibilisation du public sur les enjeux du système sera alors nécessaire pour en comprendre l’utilité. Cela exigera aussi un accès aisé aux informations via des interfaces multiples et simples (smartphones, ordinateurs, etc.).

Au niveau politique, la Plateforme Technologique de l’Union européenne finance le développement des réseaux intelligents. Aux Etats-Unis, 4,5 milliards de dollars ont été investis dans la modernisation des réseaux prévue par l’American Recovery and Reinvestment Act de 2009.