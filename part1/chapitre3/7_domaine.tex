\section{Internet of Things : des applications pour tous ?}
Il existe de nombreuses applications de l’internet des objets. Cela va de l’IoT pour les industries de la grande consommation et de l’IoT entreprise à l’IoT manufacturier ainsi qu’à l’IoT industriel (IIoT). Ces applications couvrent de nombreux secteurs verticaux, notamment l’automobile, les télécommunications et l’énergie.

\subsubsection{1. Particuliers, bâtiments intelligents et sécurité publique}

Dans le segment des consommateurs, on peut citer les maisons intelligentes avec la domotique. Elles sont équipées de thermostats, d’appareils électroménagers, de chauffage, d’éclairage ou encore d’appareils électroniques intelligents. Ils peuvent tous être connectés et commandés à distance, via des ordinateurs, smartphones et autres appareils mobiles. Les bâtiments intelligents peuvent même réduire les coûts énergétiques grâce à des capteurs qui détectent le nombre d’occupants d’une pièce.

Du côté de la sécurité publique, des dispositifs portatifs permettent d’améliorer les délais d’intervention des permiers secours, en cas d’urgence, lors d’incendie par exemple ou lors d’une crise cardiaque d’une personne portant in pace maker. ou encore grâce à des itinéraires optimisés pour l’intervention des policiers ou du SAMU. Ou encore en suivant les signes vitaux des travailleurs de chantier ou des pompiers, sur des sites où leur vie est en danger. Dans le domaine de la santé, l’IoT permet de suivre les patients de plus près.

\subsubsection{2. Hôpitaux, smart cities et entreprises}

Les hôpitaux les utilisent aussi pour la gestion des stocks de produits pharmaceutiques et les instruments médicaux. En agriculture, l’Internet des Objets permet, par exemple, de surveiller la luminosité, la température, le taux d’humidité dans l’air et dans les sols des champs cultivés. Dans une smart city, ou ville intelligente, l’IoT se déploie à travers les réverbère intelligents et des compteurs intelligents. Ils permettent notamment de réduire la circulation et améliorer l’assainissement. Mais aussi réaliser des économies d’énergie et répondre aux préoccupations environnementales.

Pour les organisations, l’Internet des objets offre de multiples avantages, comme améliorer l’expérience client. Mais aussi, surveiller l’ensemble de leurs processus opérationnels, intégrer et adapter des modèles commerciaux et prendre de meilleures décisions. Ou encore améliorer la productivité des employés, économiser du temps et de l’argent, et générer plus de revenus. Globalement, l’IoT encourage les entreprises à repenser la manière dont elles abordent leurs activités, leurs industries et leurs marchés. Et il leur donne les outils nécessaires pour améliorer leurs stratégies commerciales.
