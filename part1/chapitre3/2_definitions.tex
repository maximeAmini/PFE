\section{Définitions}

\begin{description}
	\item[Définition 1:]« Le Cloud », littéralement le nuage, est un terme hérité du jargon technique. Aux débuts d'Internet, les diagrammes techniques représentaient souvent les serveurs et l'infrastructure réseau qui composent Internet sous la forme de nuages. Alors que de plus en plus de processus informatiques étaient déplacés vers cette partie 'serveurs et infrastructures' d'Internet, l'expression « passer dans le nuage » était une manière abrégée de désigner l'endroit où les processus informatiques se déroulaient. Aujourd'hui, « le cloud » est un terme largement accepté pour ce type d'accès.
	\item[Définition 2:] Le cloud computing est la fourniture de services informatiques notamment des serveurs, du stockage, des bases de données, la gestion réseau, des logiciels, des outils d'analyse, l'intelligence artificielle,  via Internet (le cloud) dans le but d'offrir une innovation plus rapide, des ressources flexibles et des économies d'échelle. 
	\item[Définition 3:] Un paradigme de calcul distribué émergeant dans lequel les données et les services
sont disponibles dans des data centers extensibles et peuvent être accédés de manière transparente depuis des appareils (ordinateurs, téléphones, grappes, ...) connectés par Internet.
\end{description}

De manière générale, on parle de Cloud Computing lorsqu’il est possible d’accéder à des données ou à des programmes depuis internet, ou tout du moins lorsque ces données sont synchronisées avec d’autres informations sur internet. Il suffit donc pour y accéder de bénéficier d’une connexion internet.