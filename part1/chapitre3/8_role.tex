\section{Le rôle de l’Internet des objets dans le Big Data}
A mesure que le nombre d’objets connectés augmente, le volume de données générées par l’internet des objets explose. Ainsi, pour pouvoir les prendre en charge et les analyser en temps réel, il est nécessaire de s’en remettre aux outils analytiques Big Data.

Ces outils ont la capacité de traiter rapidement les larges volumes de données générées en continu par les appareils IoT, et d’en extraire des insights exploitables. Le machine learning permet notamment de repérer des modèles de données. Avec ces patterns, une entreprise peut notamment mettre en place la maintenance prédictive sur ses machines industrielles.

\textbf{Exemples de cas d’usage:}

\textit{Pour illustrer la corrélation entre IoT et Big Data, on peut prendre l’exemple des sociétés de transport. Ces dernières utilisent les données collectées par des capteurs et les outils d’analyse Big Data pour améliorer leur efficience, économiser de l’argent, et réduire leur impact sur l’environnement.}

\textit{Ou les véhicules de livraison embarquent des capteurs qui permettent de surveiller l’état du moteur, le nombre d’arrêts, la vitesse de déplacement, le nombre de kilomètres parcourus ou encore la quantité de carburant consommée.}
