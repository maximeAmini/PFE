\section{Caractéristiques du Cloud}
Cette technologie offre plusieurs caractéristiques qui sont très avantageuse pour les utilisateurs professionnels et les utilisateurs finaux. Selon le NIST, le cloud computing doit posséder 5 caractéristiques essentielles qui sont: 

\begin{itemize}[label=\textbullet]
\item \textbf{Accès réseau universel :} L'ensemble des ressources doit être accessible et à disposition de l'utilisateur universellement et simplement à travers le réseau. 

\item \textbf{Libre service à la demande :} Permet à l'utilisateur d'utiliser et de libérer des ressources distantes en temps réel en fonction des ses besoins, sans nécessiter d'intervention humaine du côté fournisseur.

\item \textbf{Ressources partagées :} Les ressources matérielles du fournisseur sont partagées entre les différents utilisateurs.

\item \textbf{Élasticité :} Les ressources allouées aux utilisateurs peuvent être augmentées
ou diminuées selon l'usage.

\item \textbf{Service mesurable et facturable (pay-as-you-use) :} Les utilisateurs
paieront pour les ressources qu'ils ont utilisées et pour la durée de leur utilisation.
\end{itemize}

\section{inconvénients du Cloud}
Cette technologie offre plusieurs avantages et bénéfices pour les utilisateurs professionnels et les utilisateurs finaux. Les trois principaux avantages sont l’approvisionnement en libre-service, l’élasticité, et le paiement à l’utilisation.

Pour de nombreuses personnes, le stockage local utilisé pendant les dernières décennies demeure aujourd’hui supérieur au Cloud Computing. Ces personnes considèrent qu’un disque dur permet de garder les données et les programmes physiquement proches, autorisant un accès rapide et simplifié pour les utilisateurs de l’ordinateur ou du réseau local.

\subsection*{Faire confiance aux opérateurs}

C’est le principal reproche émis à l’égard du Cloud. Les télécoms, les entreprises de médias et les FAI contrôlent l’accès. Faire entièrement confiance au Cloud signifie également croire en un accès continu aux données sans aucun problème sur le long terme. Un tel confort est envisageable, mais son coût est élevé. De plus, ce prix continuera d’augmenter à mesure que les fournisseurs de Cloud trouvent un moyen de faire payer plus cher en mesurant par exemple l’utilisation du service. Le tarif augmente proportionnellement à la bande passante utilisée.

En dehors de ce problème de confiance, de nombreux autres arguments s’opposent au Cloud Computing. Le cofondateur d’Apple, Steve Wozniak, a ainsi critiqué le Cloud en 2012 en présageant de nombreux problèmes de grande envergure dans les cinq années à venir. On peut par exemple redouter des crashs. Durant l’été 2012, Amazon a rencontré ce type de problème. En tant que fournisseur d’entreprises comme Netflix ou Pinterest, l’entreprise américaine a ainsi provoqué la mise hors service des plateformes de ces clients. En 2014, Dropbox, Gmail, Basecamp, Adobe, Evernote, iCloud et Microsoft ont rencontré des problèmes similaires. En 2015, ce fut le tour de Apple, Verizon, Microsoft, AOL, Level 3, Google et Microsoft. Ces désagréments ne durent généralement que quelques heures, mais représentent une perte d’argent colossale pour les entreprises affectées.

\subsection*{La question de la propriété intellectuelle}
Par ailleurs, Wozniak a exprimé ses inquiétudes concernant la propriété intellectuelle. Il est en effet difficile de déterminer à qui appartiennent les données stockées sur internet. On peut prendre pour exemple les nombreuses controverses survenues au sujet des changements de conditions d’utilisation de sites dérivés du Cloud comme Facebook ou Instagram. Ces réseaux sociaux créent la polémique en s’octroyant des droits sur les photos stockées sur leurs plateformes. Il y a également une différence entre les données mises en ligne et les données créées directement au sein du Cloud. Un fournisseur pourrait aisément revendiquer la propriété de ces dernières. La propriété est donc un facteur à prendre en compte.

Aucune autorité centrale ne gouverne l’usage du Cloud pour le stockage et les services. L’Institute of Electrical and Electronics Engineers (IEEE) tente de devenir cet organe régulateur. En 2011, il a créé l’IEEE Cloud Computing Initiative, visant à établir des standards pour l’utilisation, particulièrement dans le domaine des entreprises. Pour l’heure, les règles sont encore floues et les problèmes se règlent au cas par cas.