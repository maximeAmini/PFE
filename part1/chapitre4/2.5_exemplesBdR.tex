\subsection{Exemples de bases de données relationnelles}
Les systèmes de gestion de bases de données relationnelles (SGBDR) les plus couramment utilisés sont donnés comme suit :

\begin{itemize}[label=\textbullet]
\item \textbf{Db2 :} Est l'un des SGBD relationnelles propriétaire d'IBM 3 disponible aux utilisateurs sous licence commerciale.
\item \textbf{Microsoft SQL Server :} Le système de gestion de base de données de Microsoft 4 en langage SQL est disponible sous une licence par utilisateur payante.
\item \textbf{MySQL :} Est le SGBDR open source 5 le plus utilisé dans le monde. Depuis son acquisition par Oracle, MySQL est commercialisé sous une double licence. La communauté des développeurs d'origine poursuit le projet sous le nom de MariaDB.
\item \textbf{PostgreSQL :} avec PostgreSQL, les utilisateurs peuvent accéder gratuitement à un système de gestion de base de données relationnel-objet (SGBDRO). Le développement ultérieur est effectué par une communauté open source.
\item \textbf{Oracle Database :} le système de gestion de base de données relationnelle de la société du même nom Oracle 6 est commercialisé sous licence propriétaire contre rémunération.
\item \textbf{SQLite :} est une bibliothèque appartenant au domaine public contenant un système de gestion de bases de données relationnelles.
\end{itemize}

\textbf{Résumé :} Le schéma relationnel des bases de données est clair, mathématiquement solide et a fait ses preuves dans la pratique depuis plus de 40 ans. Pourtant, le stockage des données dans des tables structurées ne répond pas à toutes les exigences des technologies modernes de l'information.