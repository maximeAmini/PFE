\subsection{Définition : }
En informatique et en bases de données, NoSQL désigne une famille de systèmes de gestion de base de données (SGBD) qui s'écarte du paradigme classique des bases relationnelles. L'explicitation la plus populaire de l'acronyme est Not only SQL (« pas seulement SQL » en anglais) et c'est en effet ce que ce modèle de base
de données veut être : non pas une contrepartie, mais bien un enrichissement et complément utile des bases de données SQL relationnelles traditionnelles. Ce faisant, les bases de données NoSQL dépassent les limites des systèmes relationnels et exploitent un modèle de base de données alternatif. Cela ne veut toutefois pas dire qu'aucun système SQL n'est utilisé. Il existe de nombreuses variantes combinées au sein desquelles les deux solutions peuvent être utilisées et qui restent toutefois englobées sous l'étiquette NoSQL.

La définition exacte de la famille des SGBD NoSQL reste sujette à débat. Le terme se rattache autant à des caractéristiques techniques qu'à une génération historique de SGBD qui a émergé autour des années 2010. D'après Pramod J. Sadalage et Martin Fowler, la raison principale de l'émergence et de l'adoption des SGBD NoSQL serait le développement des centres de données et la nécessité de posséder un paradigme de bases de données adapté à ce modèle d'infrastructure matérielle.
\begin{itemize}[label=\textbullet]
\item \textbf{Distinction des autres termes :}
\begin{itemize}[label=\textbullet]
\item \textbf{Non SQL :} Le terme « bases de données non SQL » est trompeur, pour ne pas dire faux. Il prédit que ces bases de données sont sans aucune utilisation de SQL et ce n'est pas toujours le cas dans NoSQL. Le terme bases de données non SQL existe, mais il est plutôt utilisé comme un terme vague que comme une expression professionnelle. Cela signifie que les données sont traitées par un autre langage que SQL, par ex. XQuery pour les bases de données XML.
\item \textbf{Distributed storage :} Liés à NoSQL, il existe des termes tels que « stockage distribué » ou « stockage structuré distribué ». Il est assez difficile de distinguer ces termes de NoSQL, car ils décrivent exactement une caractéristique de NoSQL. Le stockage distribué est un terme générique pour un système qui prétend être un stockage unique mais qui est en réalité une collection de nombreuses unités informatiques stockant des parties des fichiers. Certaines bases de données NoSQL prétendent être un système de stockage distribué, par exemple BigTable de Google.
\end{itemize}
\end{itemize}