\section*{Introduction}
De nos jours, l’ubiquité de la connexion Internet est une réalité (les voitures que nous conduisons, les montres que nous portons, nos petits appareils médicaux domestiques, nos réfrigérateurs et congélateurs, nos Smartphones et ordinateurs portables). De plus, les données numériques produites par les êtres humains, dont les séquences vidéo, les photos et autres, atteignent des volumes importants de plusieurs EO par jour. 

Ces données actuellement stockées dans des bases qui leur ont été conçues spécifiquement sont gérés par des logiciels de gestion de bases de données volumineuses, jouant le rôle d’intermédiaires entre les bases de données d’un côté et les applicatifs et leurs utilisateurs de l’autre. On parle ici des bases de données non-relationnelles, dites NoSQL.
