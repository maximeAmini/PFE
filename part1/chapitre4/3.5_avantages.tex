\subsection{Les avantages NoSQL : }
Les bases de données NoSQL ont été créées en réponse aux limitations de la technologie de base de données relationnelle. Comparées aux bases de données relationnelles, les bases de données NoSQL sont plus évolutives et offrent des performances supérieures, et leur modèle de données corrige plusieurs faiblesses du modèle relationnel. Les avantages de NoSQL sont notamment :

\begin{itemize}
\item \textbf{Gros volume de données « Big data » :} le NoSql est capable de gérer un volume important de données structurées, semi-structurées et non structurées, en effet il est devenu quasi impossible, pour un unique serveur de base de données relationnelle, de répondre aux exigences des entreprises en terme de performance. Aujourd’hui, ces gros volumes de données ne sont plus un problème pour les SGBD de type NoSQL, même le plus grand des SGBD relationnel ne peut rivaliser avec une base NoSQL.

\item \textbf{Rapidité :} NoSQL n'est pas relationnelle. Pas de schéma de bases avec les contraintes sur les champs. Cela apporte de la flexibilité dans la gestion des données et la rapidité.

\item \textbf{Modèle de données flexible :} Changer le modèle de données est une vraie prise de tête dans une base de données relationnelle en production. Les systèmes NoSQL sont plus souples en termes de modèles de données, comme dans les catégories clé/valeur et documentaire. Même les modèles un peu plus stricts comme dans la catégorie orientée colonne permettent d’ajouter une colonne sans trop de problème.

\item \textbf{Solution économique :} Les bases de données NoSQL ont tendance à utiliser des serveurs bas de gamme dont le coût est moindre afin d’équiper les « clusters », tandis que les SGBD relationnels, eux, tendent à utiliser des serveurs ultra puissants dont le coût est extrêmement élevé. De ce fait, les systèmes NoSQL permettent de revoir à la baisse les coûts d’une entreprise en termes de gigabytes ou de transactions par seconde. Cela permet de stocker ainsi que de manipuler plus d’informations à un coût nettement inférieur.
\end{itemize}