\section{Les métiers du Big Data:}
Le marché du big data est à l'origine d'un nombre croissant de métiers Les entreprises de tous les secteurs cherchent désormais à exploiter les données à leur disposition pour aiguiller leur stratégie et leur développement. Toutefois, pour être en mesure d'exploiter ces données, les entreprises doivent s'appuyer sur des compétences et du savoir-faire de professionnels hautement qualifiés capables d'utiliser les technologies analytiques. Ainsi, le Big Data a donné naissance à de nombreux nouveaux métiers : 

\newcounter{compteurM}
\stepcounter{compteurM}
\subsection*{\arabic{compteurM}. LE CHIEF DATA OFFICER}
Le chief data officer se charge de gouverner la data, qui constitue un capital vital pour l'entreprise. Il a pour mission de trier les masses de données disponibles afin de faciliter l'accès à l'information pertinente permettant des prises de décision adaptées. Pour ce faire, il doit constamment vérifier la fiabilité des informations recueillies et s'appuyer sur des éléments objectifs provenant de données statistiques. Le chief data officer intervient dans la construction et la mise en application d'une stratégie de gouvernance des données et travaille en collaboration avec d'autres professionnels tels que les data scientists, les spécialistes en business intelligence et les statisticiens.

\stepcounter{compteurM}
\subsection*{\arabic{compteurM}. LE DATA ENGINEER}
Le data engineer (ou ingénieur de données) est un professionnel spécialisé dans la gestion des données. Sa mission principale consiste à recueillir, croiser, trier et réaliser des opérations de nettoyage des données. Il doit aussi gérer leur stockage dans différentes bases de données et exploiter des masses d'information sous divers formats.

\stepcounter{compteurM}
\subsection*{\arabic{compteurM}. LE DATA SCIENTIST}
Le data scientist, appelé également analyste en big data, est un spécialiste de l'analyse des données massives. Sa mission prend effet après celle de l'ingénieur de données : le data engineer intervient dans la gestion des données alors que le data scientist assure le traitement de ces données pour en extraire de la valeur. Pour ce faire, le data scientist se charge de développer des algorithmes statistiques afin de tirer des informations pertinentes permettant de classifier des données, d'anticiper un comportement ou encore de préconiser des actions appropriées. Il doit donc avoir de solides connaissances en informatique, en statistiques et en management. Le data scientist doit également maîtriser les techniques du datamining et les outils de traitement des bases de données tels que Hadoop, MapReduce, Java, BigTable et NoSQL.

Les analystes en big data interviennent dans divers domaines d’activité : ils développent dans l’e-commerce et les réseaux sociaux les algorithmes de recommandation de pages, de profils et de produits

\stepcounter{compteurM}
\subsection*{\arabic{compteurM}. L’ARCHITECTE BIG DATA}
Data architect en anglais, l'architecte big data a pour mission principale d'organiser des données brutes. C'est un métier plus conceptuel que technique, qui assure la création des infrastructures de stockage et conçoit des solutions de gestion des données massives. Il propose également aux décideurs la cartographie des outils Hadoop à mettre en place. L'architecte big data travaille en étroite collaboration avec le data scientist, tout en lui fournissant les données brutes à traiter. Il intervient également dans l'étude de la faisabilité technique et la mise en place des outils et la configuration des machines.

\stepcounter{compteurM}
\subsection*{\arabic{compteurM}. LE DÉVELOPPEUR BIG DATA}
Le développeur big data maîtrise les différents langages informatiques notamment Java et Python. Il assure la cohérence du système, la gestion des pannes et garantit la continuité du service. Les données massives sont en effet au centre des préoccupations du métier de développeur big data. Ce profil travaille aussi en collaboration avec le data scientist : alors que ce dernier intervient dans la conception des algorithmes facilitant la prise de décision, le développeur big data assure leur mise en marche. Il fait partie des rares profils du big data à pouvoir gérer toutes les catégories des outils d'Hadoop pour des objectifs d'évaluation.

\stepcounter{compteurM}
\subsection*{\arabic{compteurM}. LE GROWTH HACKER}
Le growth hacker n'est pas simplement un métier, c'est surtout un état d'esprit qui permet de développer plusieurs techniques webmarketing. C'est un profil à la croisée du marketing, du développement logiciel et du big data, qui a pour mission d'accélérer la croissance d'un produit ou d'un service propre à la structure qui l'embauche. Il utilise pour cela des solutions digitales innovantes et des pratiques de pointe afin d'accroître le revenu de son entreprise. Pour ce faire, le growth hacker cherche à développer, à partir d'Hadoop, de nouveaux produits et de nouvelles fonctionnalités. Il utilise également les outils de base de données (SQL) et les langages d'abstraction. De plus, comme tous les professionnels du marketing, il est en recherche constante de clients. Le growth hacker est très prisé par les start-up et les entreprises qui souhaitent se réinventer constamment. 

\stepcounter{compteurM}
\subsection*{\arabic{compteurM}. LE DATA MINER}
Le data miner assure la transmission des connaissances utiles à la progression de l'entreprise. Il dégage ainsi les tendances relatives à la consommation des clients pour en sortir avec une stratégie marketing réalisable sur le terrain. Pour se positionner sur le marché, il prend en compte les habitudes de consommation et les tarifs appliqués par la concurrence. De plus, il assure le tri des informations potentiellement exploitables, analyse les données après les avoir formatées et nettoyées. Il réalise aussi des rapports d'analyse, des tableaux de visualisation des données et compare les performances de l'entreprise pour les ajuster aux objectifs et prévisions. Le data miner a de grandes capacités d'observation et d'analyse de données.

\stepcounter{compteurM}
\subsection*{\arabic{compteurM}. L'ADMINISTRATEUR BIG DATA}
L'administrateur joue un rôle primordial dans la structure informatique d'une entreprise. Il intervient dans la conception, l'optimisation et la configuration des infrastructures de stockage des données massives. Il assure également la sécurisation des données ainsi que l'attribution des autorisations et des droits d'accès aux différents utilisateurs. L'administrateur big data maîtrise les langages de programmation, les outils d'administration Hadoop et les protocoles de sécurité. Il vérifie la disponibilité de l'information à tout moment et apporte les modifications nécessaires sur les bases de données.

\cite{jvc_les_2020}