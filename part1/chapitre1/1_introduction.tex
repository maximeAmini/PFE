\section*{Introduction}
Avec la mise en place des services en ligne grâce à l'utilisation extensive d'Internet, le nombre de données générées qui transitent chaque jour sur le web, n'a fait que s'accroître de manière exponentielle, on parle ici de plus de 2,5 trillions d'octets générés quotidiennement, soit plus de 29.000 Giga-octets (Go) d'informations qui sont publiées dans le monde chaque seconde.

Ses données qui sont non pas que volumineuses mais aussi hétérogènes, viennent de toute part, la majeure partie de ses dernières proviennent de trois sources principales: les données sociales(les likes, les commentaires, les tweets, les photos/vidéos…etc), les données machines (Les capteurs tels que les appareils médicaux, les caméras routières, les satellites, les jeux et l'Internet des objets fournissent des données à haute vitesse, valeur, volume et variété) et les données transactionnelles (générées à partir de toutes les transactions quotidiennes qui ont lieu à la fois en ligne et hors ligne. Les factures, les ordres de paiement, les enregistrements de stockage, les reçus de livraison …etc).

Ses données qui sont utilisées par prêt de 6 milliards d'individus chaque jour, doivent être capturées, analysées,  stockées, recherchées,  partagées, visualisées, et transférées tout cela sans atteinte à la vie privée des utilisateurs, ce qui a poussé les chercheurs à trouver de nouvelles manières de réaliser tout cela étant donné que les outils traditionnels tels que le système de gestion de base de données relationnelles (SGBDR) et le SQL se retrouve dans l'incapacité de gérer ce nombre important et hétérogènes de données, et c'est ainsi qu'est né le \textbf{“Big Data“}.

En effet, comme chaque domaine de connaissance, la terminologie naissante “Big Data“ et la science des données sont utilisées pour parler de ce phénomène, Nous allons lors de ce chapitre présenter les concepts et les définitions se rapportant au domaine du “Big Data“ quelques statistiques ainsi que, les intérêts, contraintes et caractéristiques de ce dernier.