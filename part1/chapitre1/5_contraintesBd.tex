\section{Les contraintes du Big Data :}

L'intérêt du Big Data, c'est de pouvoir tirer profit de nouvelles données produites par tous les acteurs (les entreprises, les particuliers, les scientifiques et les institutions publiques) dans le but d'optimiser son offre commerciale, ses services, développer la recherche et le développement mais aussi créer des emplois. Il y a certes des avantages mais aussi des inconvénients du Big Data.

Certaines publications discutent des obstacles au développement d'applications de méga données. Les principaux défis sont énumérés comme suit :

\begin{enumerate}[label=\protect\ding{\value*}, start=182]
\item \textbf{Représentation des données :} De nombreux ensembles de données présentent certains niveaux d'hétérogénéité dans le type, la structure, la sémantique, l'organisation, la granularité et l'accessibilité. La représentation des données vise à rendre les données plus significatives pour l'analyse informatique et l'interprétation des utilisateurs. Néanmoins, une représentation incorrecte des données réduira la valeur des données originales et peut même empêcher une analyse efficace des données.
\item \textbf{Réduction de la redondance et compression des données :} En général, il existe un niveau élevé de redondance dans les jeux de données. La réduction de la redondance et la compression des données sont efficaces pour réduire le coût indirect de l'ensemble du système en partant du principe que les valeurs potentielles des données ne sont pas affectées. Par exemple, la plupart des données générées par les réseaux de capteurs sont hautement redondantes.
\item \textbf{Gestion du cycle de vie des données :} Par rapport aux progrès relativement lents des systèmes de stockage, la détection et le calcul omniprésents génèrent des données à des taux et des échelles sans précédent. Nous sommes confrontés à de nombreux défis urgents, dont l'un est que le système de stockage actuel ne peut pas supporter des données aussi massives. De manière générale, les valeurs cachées dans le Big Data dépendent de la fraîcheur des données.
\item \textbf{Mécanisme analytique :} Le système analytique des méga données traitera des masses de données hétérogènes dans un temps limité. Cependant, les SGBDR traditionnels sont strictement conçus avec un manque d'évolutivité et d'extensibilité, ce qui ne pourrait pas répondre aux exigences de performance. Les bases de données non relationnelles ont montré leurs avantages uniques dans le traitement des données non structurées et ont commencé à se généraliser dans l'analyse des méga données. Même ainsi, il existe encore quelques problèmes de bases de données non relationnelles dans leurs performances et applications particulières. Des recherches supplémentaires sont nécessaires sur la base de données en mémoire et des échantillons de données basés sur une analyse approximative.
\item \textbf{Confidentialité des données :} La plupart des fournisseurs ou propriétaires de services de méga données ne pouvaient actuellement pas maintenir et analyser efficacement des ensembles de données aussi énormes en raison de leur capacité limitée. Ils doivent s'appuyer sur des professionnels ou des outils pour analyser ces données, ce qui augmente les risques potentiels pour la sécurité. Par exemple, l'ensemble de données transactionnelles comprend généralement un ensemble de données d'exploitation complètes pour piloter les processus métier clés. Ces données contiennent des détails et certaines informations sensibles telles que les numéros de carte de crédit.
\item \textbf{Gestion de l'énergie :} La consommation d'énergie des systèmes informatiques a beaucoup attiré l'attention du point de vue économique et environnemental. Avec l'augmentation du volume de données et des demandes analytiques, le traitement, le stockage et la transmission de données massives consommeront inévitablement de plus en plus d'énergie électrique.
\item \textbf{Expendabilité et évolutivité :} Le système analytique du Big Data doit prendre en charge les ensembles de données présents et futurs. L'algorithme analytique doit être capable de traiter des ensembles de données de plus en plus étendus et plus complexes.
\item \textbf{Coopération :} L'analyse du Big Data est une recherche interdisciplinaire, qui nécessite la coopération d'experts dans différents domaines pour exploiter le potentiel du Big Data. Une architecture de réseau Big Data complète doit être mise en place pour aider les scientifiques et les ingénieurs dans divers domaines à accéder à différents types de données et à utiliser pleinement leur expertise, afin de coopérer pour atteindre les objectifs analytiques.

\end{enumerate}
