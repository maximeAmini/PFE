\section{Définition :}

\textbf{Définition 1 :} L’IoT (Internet of Things, pour Internet des Objets) est un système d’interconnexion entre des dispositifs informatiques, des machines, des objets, des animaux et même des personnes, munies d’identifiants uniques (UID) avec la capacité de transférer des données sur un réseau. Et ce, sans interaction d’humain à humain ou d’humain à ordinateur. 

Globalement, il s’agit de tout « objet » naturel ou artificiel auquel on peut attribuer une adresse IP et qui peut transférer des données sur un réseau. De plus en plus d’entreprises, quelque soit le secteur, utilisent l’IoT pour fonctionner plus efficacement et mieux comprendre leurs clients, afin d’offrir de meilleurs services. Mais aussi pour améliorer la prise de décision et accroître la valeur de l’entreprise.

\textbf{Définition 2:} L’internet des objets peut aussi être défini selon l’UIT, comme étant une infrastructure mondiale pour la société de l’information, qui permet de disposer de services évolués en interconnectant des objets (physiques ou virtuels) grâce aux technologies de l’information et de la communication interopérables existantes ou en évolution.