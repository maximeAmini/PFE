\section{Langages et outils de développement}
Pour développer notre propre simulateur on utilise les outils et les langages de programmation suivant :

\subsection{Javascript}
JavaScript est un langage de programmation qui permet d’implémenter des mécanismes complexes sur une page web. À chaque fois qu’une page web fait plus que simplement afficher du contenu statique — afficher du contenu mis à jour à des temps déterminés, des cartes interactives, des animations 2D/3D, des menus vidéo défilants, etc... 
\\JavaScript a de bonnes chances d’être impliqué. C’est la troisième couche des technologies standards du web, les deux premières (HTML et CSS, Les trois couches se superposent naturellement. 

\subsection{vue}
Vue.JS, ou simplement Vue, est un framework progressif pour les interfaces utilisateur pour les apps et sites JavaScript. Il s’agit d’un des frameworks front-end JS les plus populaires. On le compare souvent à React, Angular, Ember, etc. Par leur approche et leur ressemblance, Vue et React partagent de nombreux points communs. Le framework apparaît à l’été 2013. Il est développé par Evan You. Peu à peu, Vue va faire parler de lui et s’imposer chez les développeurs JS.

\subsection{nodes}
NodeJS est un outil libre codé en Javascript et orientée pour des applications en réseau. Si vous êtes sur cette page, c’est certainement parce que vous voulez avoir des explications plus détaillées sur NodeJS. Cet outil JavaScript est devenu célèbre dans l’univers du développement web depuis quelques années. D’ailleurs, il est très apprécié des géants du web comme Netflix, PayPal, LinkedIn, Uber, la NASA, etc. Cet article basé sur la définition de NodeJS se donne pour rôle de vous faire découvrir de long en large cette technologie. Vous y trouverez également tous les avantages liés à son utilisation.

\subsection{express js}
ExpressJS est un framework qui se veut minimaliste. Très léger, il apporte peu de surcouches pour garder des performances optimales et une exécution rapide. Express ne fournit que des fonctionnalités d’application web (et mobile) fondamentales, mais celles-ci sont extrêmement robustes et ne prennent pas le dessus sur les fonctionnalités natives de NodeJS. 
